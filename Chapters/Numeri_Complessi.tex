% !TeX encoding = UTF-8
\chapter{Numeri complessi: cosa sono, perchè sono importanti e prime proprietà}

Numeri complessi, che saranno mai? Sono numeri complicati? Beh, il nome non è certo dovuto a quest'ultimo fatto, però ovviamente comportano alcune complicazioni rispetto ai numeri reali (ma anche molte possibilità in più :) ).


Chiaramente non sono riuscito a comprimere in un articolo (moderatamente lungo) tutto ciò che c'è da sapere sui numeri complessi. Fra qualche giorno pubblicherò la seconda parte.

\section{Introduzione ai numeri complessi}

Partiamo con una definizione, che può però dire tutto e niente:

\begin{mydef}
	Chiamiamo numero complesso ogni coppia ordinata $ (a,b) $ di numeri reali.
\end{mydef}

Intuitivamente possiamo quindi vedere ogni punto sul piano cartesiano come numero complesso. Questa è solo una delle modalità con cui questa "classe" di numeri possono essere rappresentati.
Ovviamente, come ogni buona definizione di "oggetto" matematico, essa permette di eseguire operazioni tra "oggetti" simili. Siano per esempio date le coppie $ (2,3) $ e $ (3,7) $.

Possiamo quindi introdurre il concetto di somma tra numeri complessi (nella forma di coppia ordinata) come somma componente per componente:
\[(2,3) + (3,7) = (5,10). \] 
Abbastanza semplice, no?

Qualche complicazione in più la si può incontrare nell'eseguire la moltiplicazione con questa notazione, vedremo poi delle rappresentazioni dei complessi che agevolano di molto questo calcolo.

In generale, date due coppie ordinate $ (a,b) $ e $ (a',b') $ si ha che

\[(a,b) \times (a',b') = (a\times a'-b\times b' , a\times b'+b\times a').\] Eccoti un esempio per chiarire un pelo questo concetto:

\[(2,3) \times (1,2) = (2*1-3*2, 2*2+3*1) = (-4, 7)\]

Perfetto, una volta viste queste prime definizioni e operazioni, spero che ti sia sorta almeno una delle seguenti domande:
\begin{itemize}
	\item Ma a cosa servono i numeri complessi? Mi sono sempre bastati i numeri reali..
	\item Ma i numeri complessi sono in qualche relazione con i numeri reali?
	\item Mi permettono di risolvere problemi nuovi?
	\item Tutto ciò che vale per gli altri numeri, vale anche per i complessi?
\end{itemize}
Non pretendo certo di rispondere a tutte queste domande, però almeno in parte sono sicuro che verrai soddisfatto. 
Partiamo quindi dal principio: Perchè è stato necessario "inventarsi" i numeri complessi?

\section{Utilità dei numeri complessi}
Il motivo basilare principale è la risoluzione di equazioni del tipo $ x^2+1=0 $. Ragionando in termini di soluzioni reali, siamo sempre (giustamente) stati abituati ad affermare che tale equazione sia priva di soluzioni. Il grafico della parabola $ x^2+1 $ sta infatti sempre ben sopra l'asse delle $ x $, non intersecandolo mai.

Nel campo dei numeri complessi, invece, tale equazione ha \emph{due} soluzioni. Esse sono $ (0,1) $ e $ (0,-1) $. Vedremo meglio in seguito il perchè di tali risultati. Prima è opportuno introdurre (o rinfrescare) il teorema fondamentale dell'algebra:

\begin{thm}[Teorema Fondamentale dell'algebra]
	Ogni polinomio di grado $ n>0 $ (cioè non costante), a coefficienti reali o complessi del tipo:
	\begin{equation}
	a_{n}z^{n}+\ldots +a_{1}z+a_{0}
	\end{equation}
	ammette almeno una radice complessa o zero.
\end{thm} 
Dal teorema segue che il polinomio ammette precisamente n radici complesse contate con la loro molteplicità, mentre ammette al massimo $ n $ radici reali.

In parole povere, ogni equazione di grado $ n>0 $, ha esattamente $ n $ radici. Esse possono essere complesse o reali. Vediamo un semplice esempio:
\begin{exmpl}
	$ x^3-1 = (x-1)(x^2 + x + 1) $. Essa ha chiaramente la radice reale $ x=1 $, mentre l'altro fattore in cui l'ho scomposta non ha radici reali dato che il discriminante è $ D $, con $ D = 1 - 4 = -3<0 $. Pertanto, vista la validità del teorema fondamentale dell'algebra, l'equazione qui sopra ha 1 radice reale e 2 complesse.
\end{exmpl}
Quindi l'utilità primaria dei numeri complessi è che ci permettono di trovare tutte e sole le soluzioni di un equazione di grado $ n>0 $.
\section{Numeri reali e numeri immaginari}
C'è qualche relazione tra numeri reali e numeri immaginari? Beh, certamente. Infatti vedendo le varie classi di numeri come insiemi, possiamo definire la seguente relazione di inclusione: $ \N \subset \Z \subset \Q \subset \R \subset \C $ . Dove tali lettere simboleggiano rispettivamente i numeri naturali, interi, razionali, reali e complessi.

Si può quindi chiaramente vedere che i numeri reali, sono dei particolari numeri complessi. In particolare, tutte le coppie ordinate nella forma $ (a, 0) $ con $ a\in \R $ sono corrispondenti al numero reale $ a $. Si noti infatti che tutte le operazioni tra numeri reali, come siamo abituati a vederli fin dalle medie, rimangono vere anche con le definizioni (viste al paragrafo precedente) di somma e moltiplicazione tra numeri complessi. Infatti date le coppie ordinate $ (a,0) $ e $ (b,0) $ si a che:
\begin{itemize}
	\item (a,0) + (b,0) = (a+b,0) esattamente come fare a+b dato che (a+b,0) è ad esso corrispondente
	\item $ (a,0) \times (b,0) = (a*b - 0*0, a*0 + b*0) = (ab,0) $ esattamente come fare a*b dato che (ab,0 ) è ad esso corrispondente.
	\item In contrapposizione ai numeri reali, ci sono i \emph{numeri immaginari}. Essi vengono solitamente definiti come numeri complessi del tipo $ (0,b) $ con $ b\in\R $.
	\item In particolare si definisce \emph{unità immaginaria } e la si caratterizza con il simbolo $ i $ il numero immaginario $ (0,1) $. Esso sarà fondamentale per esprimere i numeri complessi in forma algebrica.
\end{itemize}

Prima di passare oltre, ci tengo a mostrarti un calcolo molto semplice ma parecchio interessante. Facendo infatti $ i \times i = i^2 = (0,1)*(0,1) =(0*0-1*1,0*1 + 0*1) = (-1,0) = -1 $

Vedi quindi chiaramente che elevando al quadrato un numero, si ottiene un numero negativo. Parecchio strano se sei abituato a ragionare nel campo dei numeri reali, no? Infatti come ti avevo anticipato qualche riga più in su, $ i $ è una delle soluzioni dell'equazione $ x^2+1=0 $, dato che $ i^2 = -1 $ si ha che $ i^2+1=0 $ . E infatti $ i = (0,1) $ ;)

\section{La forma algebrica dei numeri complessi}
All'inizio, ci eravamo occupati di definire i numeri complessi come coppia ordinata. Da un punto di vista geometrico, tale rappresentazione è molto intuitiva. Non è però equivalentemente comoda da un punto di vista pratico della risoluzione di operazioni tra complessi. Introduciamo quindi la seconda forma in cui si possono esprimere i numeri complessi: \textbf{la forma algebrica}. Dato il numero complesso $ (a,b) $, la sua forma algebrica è semplicemente espressa come segue: $ a+bi $.

Infatti ricordato che $ i=(0,1) $, sia ha che $ a = a*(1,0) = (a,0) $ e $ b $ analogamente.
Quindi $ a+bi=(a,0) + (b,0)*(0,1) = (a,0) + (0,b) = (a,b) $.

Se per un momento chiamassimo $ n = a+bi $ , potremmo definire la parte reale di $ n $ il coefficiente $ a $ (si indica con $ \Re(n)=a $) e la sua parte immaginaria è il coefficiente $ b $ ($ \Im(n)=b $).
Bene, ti avevo detto che questa nuova modalità di esprimere i complessi era comoda per la facilità nei calcoli, giusto?
Infatti è vero, i numeri complessi possono essere sommati e moltiplicati in maniera molto naturale. Si può considerare la i come fosse una costante qualsiasi. L'unico accorgimento da avere è che la potenza dell'unità immaginaria non è proprio così immediata.

Ecco qui come calcolare le potenze n-sime dell'unità immaginaria:

potenze unità immaginaria

Quindi occhio a quando vai a moltiplicare numeri complessi tra loro in forma algebrica ;)

Ecco qui qualche esempio:
\[(5+2i) + (3+8i) = (5+3)+i(2+8) = 8 + 10i\]
\begin{align*}
	(5+2i)*(3+8i) = (5*3 + 5*8i + 2i*3 + 2i*8i ) = \\
	(15 + 40i + 6i +16(i*i))=(15 + 46i -16) = 46i - 1.
\end{align*}
Ho cercato di fare tutti i passaggi così da chiarirti un po' il meccanismo, basta farne un paio per comprendere bene come funziona ;)
\section{Confrontare numeri complessi}
Le cose introduttive relative ai numeri complessi penso di averle già dette nei paragrafi precedenti, a breve pubblicherò il seguito di questo articolo con alcuni concetti e risultati più interessanti ma leggermente meno semplici.

Prima di concludere, però, ci tengo ad introdurre una relazione d'ordine sui numeri complessi. Penso infatti che tu sappia già decidere se un numero naturale è più grande, più piccolo o equivalente ad un altro.
Bene, si può fare qualcosa di simile anche per i numeri complessi.

Dato infatti il numero $ z_1=a+bi $, definiamo modulo il numero reale positivo che si ottiene dall'operazione che segue:
\[\sqrt{a^2 + b^2}= \lvert z_1\rvert.
\]
Ora, dato un secondo complesso $ z_2=c+di  $, diciamo intuitivamente che $ z_2 $ è "più grande" di $ z_1 $, ossia $ z_2\geq z_1 $ se e solo se $ |z_2|\geq|z_1| $.

Se hai seguito con attenzione il percorso che ti ho fatto fare in questo articolo e sei a conoscenza delle nozioni base della geometria, penso che ti sia reso conto che $ |z_1| $ non descrive altro che la distanza del punto $ z_1=(a,b) $ dall'origine degli assi.

Infatti la distanza di un generico punto $ (a,b) $ del piano, rispetto all'origine, è esattamente $ \sqrt{a^2+b^2}$(teorema di Pitagora).

Prima di chiudere, un ultimo concetto: dato $ z=a+bi  $, si dice complesso coniugato di $ z $, il numero $ \bar{z} = a-bi $ , che talvolta è indicato anche con $ z^{*} $.

Per il momento è tutto, al prossimo articolo ;) 