\chapter*{Premessa}

Se trovate qualche equazione che vi sembra troppo difficile o che non capite, fate come fanno i matematici: dateci un'occhiata, cercate di coglierne il massimo possibile e saltatela semplicemente provando magari dopo qualche riga a ritornarci sopra e a ridarci una letta. La maggior parte delle volte una rotella scatterà nella vostra testa! 

% Penrose quotation
In un certo numero di luoghi in questo libro ho fatto ricorso a fornmle matematiche, senza !asciarmi spaventare dagli awerti­menti che vengono dati di solito: che ogni formula ridurrà del 50 per cento il numero dei lettori generici. Se tu sei uno dei lettori che si spaventano (come del resto accade ai più) davanti a qualsiasi formula matematica, ti raccomando un modo di procedere che normalmente adotto io stesso quando mi si presenta una riga così fastidiosa. Il procedimento consiste, più o meno, nell'ignorarla del tutto e passare semplicemente alla prossima riga di testo! Beh, non proprio così; si dovrebbe concedere alla formula se non un'attenta lettura almeno una scorsa generale, e poi passare oltre. 
Dopo un po', se ci si trova armati di nuova fiducia, si potrà tornare alla formula trascurata e cercare di coglierne qualche elemento saliente. Il testo stesso può essere utile per capire che cosa sia importante e che cosa si possa invece ignorare senza danno. Se così non è, non si abbia paura a ignorare una formula e passare
tranquillamente oltre. \cite{book:Penrose_imperatore}

